%%%%%%%%%%%%%%%%%%%%%%%%%%%%%%%%%%%%%%%%%
% Homework Assignment Article
% LaTeX Template
% Version 1.3.5r (2018-02-16)
%
% This template has been done by
% Victor Zimmermann (zimmermann@cl.uni-heidelberg.de)
%
% and edited by:
% Victoria Castor [at] vcastorv@gmail.com
% @vcastor on twitter 
% 2022 -> (2020too)
%
% License:
% CC BY-SA 4.0 (https://creativecommons.org/licenses/by-sa/4.0/)
%
%%%%%%%%%%%%%%%%%%%%%%%%%%%%%%%%%%%%%%%%%

\documentclass[a4paper,10pt]{article} % Uses article class in A4 format

%----------------------------------------------------------------------------------------
%	PACKAGES AND OTHER DOCUMENT CONFIGURATIONS
%----------------------------------------------------------------------------------------

\usepackage[a4paper, margin=2.5cm]{geometry} % Sets margin to 2.5cm for A4 Paper
\usepackage[onehalfspacing]{setspace} % Sets Spacing to 1.5
\usepackage[T1]{fontenc} % Use European encoding
\usepackage[utf8]{inputenc} % Use UTF-8 encoding
\usepackage{charter} % Use the Charter font
\usepackage{microtype} % Slightly tweak font spacing for aesthetics
\usepackage[english]{babel} %, activeacute]{babel} % Language hyphenation and typographical rules (English by defult)
\usepackage{amsthm, amsmath, amssymb} % Mathematical typesetting
\usepackage{marvosym, wasysym} % More symbols
\usepackage{float} % Improved interface for floating objects
\usepackage{graphicx, multicol} % Enhanced support for graphics
\usepackage[usenames,dvipsnames]{xcolor} % Options for fancy tables
\usepackage{rotating} % Rotation tools
\usepackage{listings, style/lstlisting} % Environment for non-formatted code, !uses style file!
\usepackage{pseudocode} % Environment for specifying algorithms in a natural way
\usepackage{style/avm} % Environment for f-structures, !uses style file!
\usepackage{booktabs} % Enhances quality of tables
\usepackage{tikz-qtree} % Easy tree drawing tool
\tikzset{every tree node/.style={align=center,anchor=north},
         level distance=2cm} % Configuration for q-trees
\usepackage{style/btree} % Configuration for b-trees and b+-trees, !uses style file!
\usepackage{titlesec} % Allows customization of titles
\renewcommand\thesection{\arabic{section}.} % Arabic numerals for the sections
\titleformat{\section}{\large}{\thesection}{1em}{}
\renewcommand\thesubsection{\roman{subsection}.} % Roman numbering for subsections
\titleformat{\subsection}{\large}{\thesubsection}{1em}{}
\renewcommand\thesubsubsection{\alph{subsubsection})} % Alphabetic numerals for subsubsections
\titleformat{\subsubsection}{\large}{\thesubsubsection}{1em}{}
\usepackage[all]{nowidow} % Removes widows

%----------------------------------------------------------------------------------------
%---- NEW packages 
%----------------------------------------------------------------------------------------

\usepackage{chemfig} % Molecule drawing
\usepackage{multirow} % Multiples row in the same page
\usepackage{wrapfig} % Allows figures or tables to have text wrapped around them
\usepackage [ all ]{xy} % Diagrams for thermodinamics
\usepackage{mathrsfs} % For fancy caligraphy \mathscr{}
\usepackage{subcaption} % subcaptions
\usepackage[version=3]{mhchem} % \ce for [H+] and that kind of things
\usepackage{adjustbox} % move and adjust boxes in and out side paper size
\usepackage{physics} % easer way to write differentials
\usepackage{breqn}
\usepackage{longtable} %
\usepackage{epsfig}
\usepackage{braket} % Dirac notation
\usepackage{ctable} % Extra configurations for tables
\usepackage{wrapfig} % Figures to the right or left 
\usepackage{siunitx} % Units, Never imperial, we're not USA
\sisetup{
group-minimum-digits = 4
%separate-uncertainty
}
\DeclareSIUnit{\calorie}{cal} % Calorie just because
\usepackage{nicefrac} % Cool way for factions as 1/2
\usepackage{xfrac} % Another cool way of 1/2
\usepackage{url} % Links to URL
\usepackage{csquotes} % Context sensitive quotation facilities
\usepackage{pdfpages} % 
\usepackage{fancyvrb} 
%\usepackage[caption = false]{subfig}

%----------------------------------------------------------------------------------------
%---- hyperref, defining colours also
\usepackage{hyperref}
\definecolor{zaffre}{rgb}{0.0, 0.08, 0.66}
\definecolor{slateblue}{rgb}{0.42, 0.35, 0.8}
\definecolor{royalazure}{rgb}{0.0, 0.22, 0.66}
\definecolor{smalt(darkpowderblue)}{rgb}{0.0, 0.2, 0.6}
\definecolor{antiquefuchsia}{rgb}{0.57, 0.36, 0.51}
\definecolor{ashgrey}{rgb}{0.7, 0.75, 0.71}
\hypersetup{
    colorlinks=true,
    linktoc=all,
    linkcolor=smalt(darkpowderblue),
    linktocpage=true,
    citecolor=antiquefuchsia,
    pdfauthor={Victoria Castor}
    pdftitle={pdf bonito <3}
}
%---- Fancy table
\usepackage{tcolorbox}
\usepackage{tabularx}
\usepackage{array}
\usepackage{colortbl}
\tcbuselibrary{skins}
\newcolumntype{Y}{>{\raggedleft\arraybackslash}X}
\tcbset{tab1/.style={fonttitle=\bfseries\large,fontupper=\normalsize\sffamily,
colback=yellow!10!white,colframe=red!75!black,colbacktitle=Salmon!40!white,
coltitle=black,center title,freelance,frame code={
\foreach \n in {north east,north west,south east,south west}
{\path [fill=red!75!black] (interior.\n) circle (3mm); };},}}
\tcbset{tab2/.style={enhanced,fonttitle=\bfseries,fontupper=\normalsize\sffamily,
colback=yellow!10!white,colframe=red!50!black,colbacktitle=Salmon!40!white,
coltitle=black,center title}}

%----------------------------------------------------------------------------------------
%---- Bibliography
\usepackage[sort&compress]{natbib}
\bibliographystyle{unsrt}
%\usepackage[square,numbers]{natbib}
%\bibliographystyle{abbrvnat}


%----------------------------------------------------------------------------------------
%---- Date and footers notes
\usepackage[yyyymmdd]{datetime} % Uses YEAR-MONTH-DAY format for dates
\renewcommand{\dateseparator}{-} % Sets dateseparator to '-'
\usepackage{fancyhdr} % Headers and footers
\pagestyle{fancy} % All pages have headers and footers
\fancyhead{}\renewcommand{\headrulewidth}{0pt} % Blank out the default header
%\fancyfoot[L]{\textsc{Ciudad de México, Estados Unidos Mexicanos}} % Custom footer text
%\fancyfoot[L]{\textsc{Madrid, Reino de España}} % for example, where you are writting
\fancyfoot[L]{\textsc{Paris, République française}} % or to where will be send the text
\fancyfoot[C]{} % Custom footer text
\fancyfoot[R]{\thepage} % Custom footer text
\newcommand{\note}[1]{\marginpar{\scriptsize \textcolor{red}{#1}}} % Enables comments in red on margin
\usepackage{lmodern} %texttt and textbf at the same time

%----------------------------------------------------------------------------------------
%---- Julia programming language isn't deffined by defult
%% Julia definition (c) 2014 Jubobs
\lstdefinelanguage{Julia}%
  {morekeywords={abstract,break,case,catch,const,continue,do,else,elseif,%
      end,export,false,for,function,immutable,import,importall,if,in,%
      macro,module,otherwise,quote,return,switch,true,try,type,typealias,%
      using,while},%
   sensitive=true,%
   alsoother={$},%
   morecomment=[l]\#,%
   morecomment=[n]{\#=}{=\#},%
   morestring=[s]{"}{"},%
   morestring=[m]{'}{'},%
}[keywords,comments,strings]%

\lstset{%
    language         = Julia,
    basicstyle       = \ttfamily,
    keywordstyle     = \bfseries\color{blue},
    stringstyle      = \color{magenta},
    commentstyle     = \color{ForestGreen},
    showstringspaces = false,
}

%---- FORMATTING, if u want
%\setlength{\parskip}{0pt}
%\setlength{\parindent}{0pt}
%\setlength{\voffset}{-15pt}
%\setlength{\parindent}{12pt}

