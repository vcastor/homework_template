%----------------------------------------------------------------------------------------
%	TITLE OF THE HOMEWORK	
%----------------------------------------------------------------------------------------
{\setlength{\parindent}{0pt}
\title{Task2} % Article title
\fancyhead[C]{}
\begin{minipage}{0.295\textwidth} % Left side of title section
\raggedright
MQT II\\ % Your lecture or course
\footnotesize % Authors text size
%\hfill\\ % Uncomment if right minipage has more lines
Victoria Castor Villegas % Your name, your matriculation number
\medskip\hrule
\end{minipage}
\begin{minipage}{0.4\textwidth} % Center of title section
\centering 
\large % Title text size
pKa determination\\ % Assignment title and number
\normalsize % Subtitle text size
Task 2\\ % Assignment subtitle
\end{minipage}
\begin{minipage}{0.295\textwidth} % Right side of title section
\raggedleft
\today\\ % Date
\footnotesize % Email text size
%\hfill\\ % Uncomment if left minipage has more lines
victoria.castor@estudiante.uam.es% Your email
\medskip\hrule
\end{minipage}
}
%----------------------------------------------------------------------------------------
%	HOMEWORK CONTENTS
%----------------------------------------------------------------------------------------

\section{\textbf{Introduction}}

In this work about the pKa determination using continuum solvation
models, we compute the expected value from theoretical point of view
for pKa of 6,7-dihydroxycoumarin-3-carboxylic acid.
To understand how compute this value, we just need know about how
the $K_a$ is related with the Gibbs free energy, that since we are
modeling the aqueous reaction as a contributions in a gas phase, as
is written down in the next Equation.

%\xymatrix@C=1cm{
%\xymatrix{
\centerline{\xymatrix@C0.6cm{
AH_{(g)}  \ar[r]^{\Delta G_{g}} & A^-_{(g)} \ar[d]^{\Delta G_{solv,A^-}}    & + & H^+_{(g)} \ar[d]^{\Delta G_{solv,H^+}}\\
AH_{(aq)} \ar[u]^{-\Delta G_{solv,AH}} \ar[r]^{\Delta G_{aq}} & A^-_{(aq)} & + & H^+_{(aq)}
}}

Therefore, if we have the values for $\Delta G$'s we can compute easily
and estimate pKa value. From Quantum Chemistry we take the models
about solvation, particularly SMD model. That to have the corresponded
numerical values to compute pKa, just following the simplest equations
taken from the pKa and $K_a$ definitions, also how the $\Delta G$ is
related with the reactions.

\begin{align}
\mathrm{pKa} = -logK_a = \frac{\Delta G\mathrm{_{aq}}}{\mathrm{RT}ln(10)}, \ with: \
K_a = \frac{[A^-_{aq}][H^+_{aq}]}{[AH_{aq}]}
\end{align}

\begin{align}
\Delta G\mathrm{_{aq}} &= \Delta G\mathrm{_{gas}} +
\Delta\Delta G\mathrm{_{solv}} \\\nonumber
\Delta G\mathrm{_{gas}} &= G\mathrm{^{\circ}_{gas,A^-}} +
G\mathrm{^{\circ}_{gas,H^+}} - G\mathrm{^{\circ}_{gas,AH}} \\\nonumber
\Delta\Delta G\mathrm{_{solv}} &= \Delta G\mathrm{_{solv,A^-}} +
\Delta G\mathrm{_{solv, H^+}} - \Delta G\mathrm{_{solv, AH}}
\end{align}

Also, as we said, we are computing these from Solvation Model Based on
Density (SMD). Published in 2009, \cite{Marenich2009}; SMD directly calculates
the solvation free energy of an ideal solvation process that occurs at fixed
concentration. That with fixing the concentration free energy of solvation into
two components. The first component is the bulk-electrostatic contribution
arising from a self-consistent reaction field (SCRF) treatment; the second one,
arising from short-range interactions between the solute and solvent molecules
in the first solvation shell.


\section{\textbf{Computational Details}}

We optimized the structures for both states of our molecule (proton and
deprotonated) with
B3LYP/6-31G(d)~\cite{Becke1993,Lee1988,Vosko1980,Stephens1994,Ditchfield1971}
approximation, with SMD \cite{Marenich2009}. Later on, a single point
calculation with the same theory level in gas phase. All these calculations
were carried out using the package {\sc{Gaussian16}}~\cite{g16}.

The starting guessed geometry for the molecule was taken from Moodle Platform. With the
help of {\sc{GaussView}}~\cite{g16} and \texttt{vi} editor we made the input
files.

Later on, we get the values from the {\sc{Gaussian16}} outputs with an simple
\texttt{grep} command, looking for the strings: \texttt{SCF Done}, and
\texttt{Gibbs Free}. Where the last value for \texttt{SCF Done} is from
the gas phase and the value before is for the solvation case. That
since we computed both cases from the same input file, extending
that with \texttt{--Link1--} reserved word.

\section{\textbf{Results}}

After the computing time (less than 4 hours with an i5 2015 processor), we had the
next values:

\begin{tcolorbox}[tab2,tabularx={X|Y|Y|Y},title=Numerical Values (a.u.), boxrule=0.5pt]
 & Equation & AH & A$^-$ \\\hline\hline
$E\mathrm{_{sol}}$  & $\langle\psi_{\mathrm{sol}} |\hat{H}^{\circ} + \frac12V_{\mathrm{sol}} |
\psi_{\mathrm{sol}}\rangle + \Delta G\mathrm{_{non-ele}}$ & -836.061420211
& -835.586374975 \\\hline
$E\mathrm{_{vac}}$ & $\langle\psi_0 |\hat{H}^{\circ} |\psi_0 \rangle$ & -836.031409140 & -836.031409140 \\\hline
$G\mathrm{_{corr}}$ &  & 0.110238 & 0.096639 \\
\end{tcolorbox}

Therefore, we can compute the value for the Gibbs free energy of solvation as:

\begin{align}
\Delta G_{\mathrm{solv}} = E_{\mathrm{sol}} - E_{\mathrm{vac}} &=
\left< \psi_{\mathrm{sol}} \left| \hat{H}^{\circ} + \frac12V_{\mathrm{sol}} \right| \psi_{\mathrm{sol}} \right> -
\langle \psi_0 | \hat{H}^{\circ}| \psi_0 \rangle
+ \Delta G_{\mathrm{non-ele}} \\
&= \Delta G_{\mathrm{ele}} + \Delta G_{\mathrm{non-ele}}\nonumber
\end{align}

\noindent Eventually, compute the Gibbs energy in solution from DFT calculation in
water,

\begin{align}
G_{\mathrm{sol}} = E_{\mathrm{vac}} + G_{\mathrm{corr}} + \Delta G_{\mathrm{solv}} =
G_{\mathrm{vac}} + \Delta G_{\mathrm{solv}}
\end{align}

Then, using all the equations written down before, we can compute the value
for $\Delta G_{\mathrm{(aq)}}$ as:

\begin{align}
\Delta G_{(aq)} =& G\mathrm{^{\circ}_{gas,A^-}} +
G\mathrm{^{\circ}_{gas,H^+}} - G\mathrm{^{\circ}_{gas,AH}} +
\Delta G\mathrm{_{solv,A^-}} +
\Delta G\mathrm{_{solv, H^+}} - \Delta G\mathrm{_{solv, AH}}\\\nonumber
=& E_{\mathrm{DFT,A^-}} + G_{\mathrm{corr,A^-}} + G^{\circ}_{\mathrm{gas,H^+}} - (E_{\mathrm{DFT,AH}} + G_{\mathrm{DFT,AH}}) + \\\nonumber
&E_{\mathrm{sol,A^-}}-E_{\mathrm{vac,A^-}} + \Delta G_{\mathrm{solv,H^+}} - (E_{\mathrm{sol,AH}}-E_{\mathrm{vac,AH}})
\end{align}

Therefore, not forgetting to convert to solution 1 M standard state by adding 1.89
kcal/mol to G$^{\circ}$, we can aim that, in these level of theory, the expected value for the
molecule $\Delta G_{(aq)}$ and the pKa at 298.15 K is:

\begin{align}
\Delta G_{(aq)} &= 297.25 +(-265.01) - (-18.83)
-524549.07 +(-6.28) -(-524540.53) \,\mathrm{kcal/mol}\\
&= 80.62\, \mathrm{kJ/mol} \\\nonumber
\mathrm{pKa} &= 14
\end{align}

\section{\textbf{Molecule}}

The deprotonation of 6,7-dihydroxycoumarin-3-carboxylic acid:

\schemestart
\chemfig{*6((-HO)=-*6(-O-(=O)-(-(=[2]O)-[:-30]O^\ominus)=--)=-=(-HO)-)} \+
\chemfig{H^\oplus}
\schemestop
